\documentclass[12pt]{article}
\usepackage[margin=1in]{geometry}
\usepackage{hyperref}
\usepackage{url}

\title{Proposal for Research and Education Projects\\
	 BHAVI 2016 Summer Education Program}
\author{Jason Liu}
\date{7/1/2016}

\begin{document}
\maketitle

\section{Primary Research Project}
	\begin{description}
		\item[Project Title:] Using MySQL with MeanStack in an Alternate Implementation of PDS
		\item[Project Authors:] Jason Liu, Daniel Yang, Adam Craig, Dr. Carl Taswell, MD, PhD 
	\end{description}

	For researchers, primary research is only the first stepping stone to proving or disproving a theory. One successful experiment proves little to nothing in the grand scheme of the scientific community. However, the weakness of a single experiment is covered by the power of the meta-analysis, usually completed by a third party. Meta-analysis requires researchers to analyze results from muliple primary articles and assess whether or not the result is accurate, show if the hypothesis presented are proven or disproven, and provide a degree of error. However, despite the necessity of meta-analyses, it continues to be time consuming and ineffective at connecting with all points of data. The miniscule amount of secondary articles pale in comparison to the literal thousands of primary articles which are published each day. In order to take advantage of all the new information which is provided and to create more effective secondary analyses, a semantic web solution provides the necessary tools in order to take full statistical advantage that meta-analysis offers.

	In order to support PORTAL registries and DOORS directories, a database is necessary in order to support the data. In order to take full advantage of a tried and true framework, Meanstack serves as the ideal system. However, the 'M,' standing for MongoDB is a NoSQL database. Although NoSQL may offer certain benefits, t a relational database may continue to be the reliable solution for the semantic web.  

\hfill

	The implementation of a database in a semantic web solution must:
\begin{itemize}
\item be a network of registries and directories for resource metadata organized by content rather than technological format
\item encourage interoperability and compatibility with other specialized communities (ie connecting clinical neuroscience with cognitive neuroscience)
\item be of hybridized architecture able to handle XML Schemas, RDF triples and OWL ontologies in order to bridge the transition between semantic web and original web smoothly. 
\item have logical hierarchical authorities and generally understood identifiers
to prevent conflictictions with names when identifying resources 
\item have the potential for greater expansion and adoption by the global scientific and public community. 
\end{itemize}

	In order to achieve these goals, numerous tools are necessary. At the core of the project, MySQL will be the primary server implementation. MySQL will be the relational database alternative for the current noSQL solution in Meanstack (MongoDB, Express, AngularJS, NodeJS). By taking advantage of this system, the project can take full advantage of all four tools under one language. However, in order to replace MongoDB with MySQL, some ORM (Object Relational Mapper) will be needed, possibly Telerik, Sequelize, etc. An ORM will also be needed to convert previous data from MicrosoftSQL to MySQL. Following primary tools, secondary tools are a necessity. Due to the nature of the project falling under multiple sections and individuals, communication with either Google Hangouts, Skype, or GoToMeetings will be used. 

\hfill


Timeline:
\begin{itemize}
	\item Week 1-2: Create read/write MySQL database from MicrosoftSQL
	\item Week 3-4: Set up PORTAL system for database
	\item Week 5-6: Set up DOORS system for database
	\item Week 7-8: Connect with other modules 
	\item Week 9-10: Test reliability, accuracy, full extent of analysis
\end{itemize}

\section{Secondary Research Project}

\begin{description}
\item[Project Title:]{Using MEAN Stack to Develop a Web App that Performs Meta-Analysis}
\item[Project Authors:] Daniel Yang, Jason Liu, Adam Craig, Dr. Carl Taswell, MD, PhD
\end{description}
	The project plans to use MEAN stack to develop a website that uses meta-analysis to reach an answer to an user inputted question after searching through all the relevant articles in a database (in this project MySQL). After reaching a general consensus, the algorithm will return the solution as a simplified paragraph that the user can read. The goal of this project is to enable researchers to be able to create reliable meta analyses without the typical time consumption and inaccuracy. My role is to simply work toward and help achieve the primary goal of the project. My primary works with my secondary project in the manner that it combines to create the front and backend of the semantic web projecet. 

\section{Software Education Project}

\begin{itemize}
\item SQL: Used for personal projects through c9.io
\item HTML5, CSS3 and JavaScript: Used for Web App Development (class) and personal projects
\item XML: Used for Android projects through AndroidStudio in Mobile App Development (class) and personal projects
\end{itemize}


\section{References}
\begin{enumerate}
\item Barrasa Rodriguez, J., Corcho, Ó. and Gómez-Pérez, A.
R2O, an extensible and semantically based database-to-ontology mapping language
Springer-Verlag, 2004
\item Berners-Lee, T., Hendler, J., Lassila, O. and others
The semantic web
Scientific american, New York, NY, USA:, 2001, Vol. 284(5), pp. 28-37
\item Dickey, J.
Write modern web apps with the MEAN stack: Mongo, Express, AngularJS, and Node. js
Pearson Education, 2014
\item MySQL, A.
MySQL
2001
\item Pan, Z. and Heflin, J.
Dldb: Extending relational databases to support semantic web queries
DTIC Document, DTIC Document, 2004
\item Spanos, D.-E., Stavrou, P. and Mitrou, N.
Bringing relational databases into the semantic web: A survey
Semantic Web, IOS Press, 2012, Vol. 3(2), pp. 169-209
\item Stojanovic, L., Stojanovic, N. and Volz, R.
Migrating data-intensive web sites into the semantic web
Proceedings of the 2002 ACM symposium on Applied computing
2002, pp. 1100-1107
\item Suehring, S.
MySQL bible
John Wiley and Sons, Inc., 2002
\item Taswell, C.
DOORS to the semantic web and grid with a PORTAL for biomedical computing
IEEE Transactions on Information Technology in Biomedicine, IEEE, 2008, Vol. 12(2), pp. 191-204
\item Taswell, C.
Portals and doors for the semantic web and grid
Google Patents, 2010
\item Taswell, C.
A distributed infrastructure for metadata about metadata: The HDMM architectural style and PORTAL-DOORS system
Future Internet, Molecular Diversity Preservation International, 2010, Vol. 2(2), pp. 156-189
\end{enumerate}


% References: note that the \nocite{*} command will automatically generate 
%  a reference list for all references contained in all *.bib files separated by
%  commas in the \bibliography{} command.
% \nocite{*}
\bibliographystyle{plain}
\bibliography{BrainWarping}
\end{document}